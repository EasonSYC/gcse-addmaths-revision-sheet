\documentclass[8pt]{article}
\usepackage{allan-eason}

\usepackage{siunitx}

\usepackage[english]{babel}

\setcounter{tocdepth}{1}

\theoremstyle{remark}

\newtheorem{theorem}{Theorem}[section]
\newtheorem{lemma}[theorem]{Lemma}
\newtheorem{corollary}[theorem]{Corollary}
\newtheorem{definition}[theorem]{Definition}
\newtheorem{example}[theorem]{Example}

\newtheorem*{remark}{Remark}

\usetikzlibrary{positioning}
\usetikzlibrary{svg.path}

\graphicspath{ {./images/} }


\newcommand{\Date}{\today}
\newcommand{\Name}{GCSE Maths Knowledge Sheet}
\newcommand{\Subname}{Eason's Mathematics Toolbox}
\newcommand{\Title}{\Name\\ \Large{\Subname}}

\newcommand{\Author}{Yicheng Shao}

\author{\Author}
\title{\Title}
\date{Version 1. \Date}

\lhead{\Name}

\begin{document}

	\maketitle

	\tableofcontents

    \section*{What is this and why this?}

        Mathematics is one of my favourite subjects and it is very important whatever you are doing in the future. It is mostly about techiniques solving problems, but the knowledge behind all those techiniques are vital for understanding. To aid practice, I produced this document based on the syllabus.

        This is more of an extension of the syllabus and the structure is exactly the same. However, it provides some sample answers for those questions in the syllabus and is a good way to refer to your self assessment based on the syllabus.

        I am also an IGCSE student so errors are inevitable in this document. Feel free to email \href{eason.syc@icloud.com}{eason.syc@icloud.com} to point out any mistakes or submit an issue on the GitHub page!

        This document assumes prior knowledge in CIE IGCSE Mathematics.

    \setcounter{section}{-1}
    
    \section{Prior Knowledge and Notations}
        \subsection{Set Notations}

            \begin{definition}[set and elements]
                If $x$ is an element of set $S$, we denote $x \in S$. Otherwise, $x \notin S$.
            \end{definition}

            \begin{definition}[set constructers]
                We have two set constructer notations:
                $$
                    \{x \mid P(x)\} = \{x : P(x)\}
                $$
                defines a set containing all $x$ satisfying condition $P(x)$.

                $$
                    \{x_1, x_2, \ldots\}
                $$
                defines a set with elements $x_1, x_2, \ldots$.
            \end{definition}

            \begin{definition}[empty set, universal set]
                We use $\emptyset$ or $\varnothing$ to define the empty set (set with no elements) and use $\mathcal{E}$ to denote the universal set.
            \end{definition}

            \begin{definition}[cardinality]
                We use $n(S)$ to denote the number of elements in set $S$.
            \end{definition}

            \begin{definition}[complement]
                We use $S'$ to define the complement of set $S$,
                $$
                    S' = \{x \in \mathcal{E} \mid x \notin S\}
                $$
            \end{definition}

            \begin{definition}[subset, proper subset]
                We denote 
                $$A \subseteq B$$
                if $x \in A \implies x \in B$.

                Furthermore, if $A \neq B$, we denote it as
                $$A \subset B.$$
            \end{definition}

            \begin{definition}[union, intersection]
                We denote
                $$
                    A \cap B = \{x \mid x \in A \lgand x \in B\},
                $$
                and
                $$
                    A \cup B = \{x \mid x \in A \lgor x \in B\}.
                $$
            \end{definition}

            \begin{definition}[number sets]
                The set $\NN$ is the natural numbers, $\NN = \{ 1, 2, 3, \ldots \}$.

                The set $\ZZ$ is the integers, $\ZZ = \{\ldots, -3, -2, -1, 0, 1, 2, 3, \ldots\}$.

                The set $\QQ$ is the rational numbers,
                $$
                    \QQ = \left\{\frac{p}{q} \mid p, q \in \ZZ, q \neq 0\right\}.
                $$

                The set $\RR$ is the set of real numbers.
            \end{definition}

            \begin{definition}[intervals]
                We define the intervals as follows:
                \begin{align*}
                    (a, b) &= \{x \mid a < x < b\},\\
                    (a, b] &= \{x \mid a < x \leq b\},\\
                    [a, b) &= \{x \mid a \leq x < b\},\\
                    [a, b] &= \{x \mid a \leq x \leq b\},\\
                    (a, +\infty) &= \{x \mid x > a\},\\
                    [a, +\infty) &= \{x \mid x \geq a\},\\
                    (-\infty, b) &= \{x \mid x < b\},\\
                    (-\infty, b] &= \{x \mid x \leq b\},\\
                    (-\infty, +\infty) &= \RR.
                \end{align*}
            \end{definition}

        \subsection{Relationship Symbols and Operations}
            \begin{definition}[implies, implied by, equivilent]
                $A$ implies $B$ is denoted by $A \implies B$, $B$ implies $A$ is denoted by $A \impliedby B$, $A$ and $B$ are equivilent is denoted by $A \iff B$.
            \end{definition}

            \begin{definition}[sum and product]
                We define
                $$
                    \sum_{i = 1}^{n} a_i = a_1 + a_2 + \ldots + a_n,
                $$
                and
                $$
                    \prod_{i = 1}^{n} a_i = a_1 \cdot a_2 \cdots a_n.
                $$
            \end{definition}

            \begin{definition}[binominal coefficient and factorial]
                We define
                $$
                    n! = n \cdot (n - 1) \cdots 1,
                $$
                with $0! = 1$, hence defining
                $$
                    \binom{n}{r} = \frac{n!}{r!(n-r)!}
                $$
            \end{definition}
        
        \subsection{Functions}
            \begin{definition}[composite of two functions]
                We define
                $$
                    gf(x) = g(f(x)).
                $$
            \end{definition}

            \begin{definition}[derivative]
                We denote
                $$
                    \frac{\diff^n f(x)}{\diff x^n} = f^{(n)} (x)
                $$
                as the $n$th derivative of $f(x)$.
            \end{definition}

        \subsection{Triangles}
            \begin{theorem}[sine rule, cosine rule, area]
                In $\triangle ABC$ with side lengths $a, b, c$ and angles $A, B, C$, we have
                $$
                    \frac{a}{\sin A} = \frac{b}{\sin B} = \frac{c}{\sin C},
                $$
                $$
                    a^2 = b^2 + c^2 - 2bc \cos A,
                $$
                $$
                    b^2 = a^2 + c^2 - 2ac \cos B,
                $$
                $$
                    c^2 = a^2 + b^2 - 2ab \cos C,
                $$
                $$
                    \text{area}=\frac{1}{2} ab \sin C = \frac{1}{2} bc \sin A = \frac{1}{2} ac \sin B.
                $$
            \end{theorem}

    \section{Functions}

        \begin{definition}[function, domain, image]
            A \textbf{fucntion} $f: A \rightarrow B$ is defined as a mapping which maps each element in $A$ to exactly one element in $B$. Basically, a function is an operation on a thing which definitely produces another thing.

            We call $A$ the \textbf{domain} (the set which this function can operate on). (And $B$ the co-domain.)

            We define the set
            $$
                \{f(x) \mid x \in A\}
            $$
            as the \textbf{range} of the function, which is all the outputs of the function.

            At this stage, $B$ will be $\RR$ and $A$ will be a subset of $\RR$.
        \end{definition}

        \begin{definition}[one-to-one, many-to-one]
            We call a function $f$ \textbf{one-to-one}, or injective, when
            $$
                f(x_1) = f(x_2) \implies x_1 = x_2.
            $$
            
            This means that each output a function will produce can only appear by opeating on exactly one element.

            If a function is not one-to-one, we call it \textbf{many-to-one}.
        \end{definition}

        \begin{definition}[function notations]
            The result that $f$ maps an element of the domain $x$ to is denoted as $f(x)$. As an example, if function $f$ maps $x$ to $\sin x$, then the following are equivilent:
            \begin{enumerate}
                \item $f(x) = \sin x$,
                \item $f: x \mapsto \sin x$.
            \end{enumerate}
        \end{definition}

        \begin{definition}[inverse]
            A function's inverse, denoted as $f^{-1}(x)$, is defined from the range of $f(x)$ to the domain of $f(x)$, and satisfies that:
            $$
                f^{-1}(f(x)) = f(f^{-1}(x))= x.
            $$
        \end{definition}

        \begin{theorem}[unique inverse]
            If the inverse of a function exists, then it is unique.
        \end{theorem}

        \begin{theorem}[condition for existance of inverse]
            The inverse of a function exists if and only if such function is one-to-one.
        \end{theorem}

        \begin{remark}
            The previous theorem is true if and only if such inverse is defined from the range. If such inverse is defined from the co-domain then we also require the function to surjective (i.e. range equals to co-domain) hence bijective. This is a very useful concept (isomorphism)!
        \end{remark}

        \begin{theorem}[inverse graphs]
            The graph of a inverse of a function and the function itself is symmetric by the line $y = x$.
        \end{theorem}

        \begin{definition}[composite]
            The composite of $f$ with $f$ denoted as $f^2$ is defined as follows:
            $$
            f^2(x) = f(f(x)).
            $$
        \end{definition}

        \begin{definition}[modulus]
            The graph of $\abs{f(x)}$ and $f(x)$ has a relationship as follows:

            The graph of $\abs{f(x)}$ reflects the part of the graph of $f(x)$ below the $x$ axis with regards to the $x$ axis (basically flip it up).
        \end{definition}

    \section{Quadratic}
        \begin{definition}[quadratic]
            A quadratic function $f$ is defined as an element of $\PP[x]$ where $\deg f(x) = 2$.

            Just kidding. A quadratic function $f$ is defined as
            $$
            f(x) = ax^2 + bx + c
            $$
            where $a \neq 0$.
        \end{definition}

        \begin{theorem}[extremum property]
            A quadratic function $f(x)$ has a maximum if and only if $a < 0$, and it has a minimum if and only if $a > 0$. The turning point (extremum point in this case) of a quadratic is
            $$
                \left(-\frac{b}{2a}, \frac{4ac - b^2}{4ac}\right).
            $$

            \begin{proof}
                We can show this by \textbf{completing the square}.
                \begin{align*}
                    ax^2 + bx + c &= a \left(x^2 + \frac{b}{a} x\right) + c\\
                    &= a \left(x^2 + 2 \cdot \frac{b}{2a} \cdot x \right) + c\\
                    &= a \left[x^2 + 2 \cdot \frac{b}{2a} \cdot x + \left(\frac{b}{2a}\right)^2\right] - a \cdot \left(\frac{b}{2a}\right)^2 + c\\
                    &= a \cdot \left(x + \frac{b}{2a}\right)^2 - \frac{b^2}{4a} + c.
                \end{align*}

                If $a > 0$, then we have
                \begin{align*}
                    ax^2 + bx + c \geq -\frac{b^2}{4a} + c,
                \end{align*}
                where the equal sign holds if and onlly if $x = - \frac{b}{2a}$.

                Similar argument holds for $a < 0$.
            \end{proof}

            \begin{proof}
                We can also show this by \textbf{differentiation}.
            \end{proof}
        \end{theorem}

        \begin{theorem}[roots]
            The roots (solutions) to the quadratic will be
            $$
                x_{1, 2} = \frac{-b \pm \sqrt{b^2 - 4ac}}{2a}.
            $$
        \end{theorem}

        \begin{theorem}[discriminant]
            The \textbf{discriminant} for the quadratic $ax^2 + bx + c$ is defined as
            $$
                \Delta = b^2 - 4ac.
            $$

            When $\Delta > 0$, the quadratic has two distinct real roots; when $\Delta = 0$, the quadratic has two equal real roots; when $\Delta < 0$, the quadratic has two complex roots which are complex conjugate of each other (i.e. sum to a real number).
        \end{theorem}

        \begin{theorem}[quadratic and a line]
            The intersections for a quadratic and a line (which is not perpendicular to the $x$ axis) can be found by equating their equations and solve the corresponding equation (which is a quadratic).
        \end{theorem}

        \begin{theorem}[quadratic inequalities]
            A quadratic inequality can be solved by finding the two solutions (known as \textbf{critical values}).

            For the quadratic $f(x) = ax^2 + bx + c$ where $a > 0$ ($a < 0$ can be considered similarly),
            \begin{enumerate}
                \item $\Delta > 0$. Let the two roots be $x_1$ and $x_2$.
                
                The solution set to $f(x) > 0$ is
                $$
                    (-\infty, x_1) \cup (x_2, +\infty).
                $$

                The solution set to $f(x) < 0$ is
                $$
                    (x_1, x_2).
                $$

                \item $\Delta = 0$. Let the root be $x_r$.
                
                The solution set to $f(x) > 0$ is
                $$
                    (-\infty, x_r) \cup (x_r, +\infty).
                $$

                The solution set to $f(x) < 0$ is $\emptyset$.

                \item $\Delta < 0$. The solution set to $f(x) > 0$ is $\RR$ and the solution set to $f(x) < 0$ is $\emptyset$.
            \end{enumerate}
        \end{theorem}

    \section{Equations, inequalities and graphs}
        \begin{theorem}[type $\abs{ax + b} = c, a \neq 0, c \geq 0$]
            The solutions to the equation
            $$
                \abs{ax + b} = c
            $$
            is
            $$
                x_1 = \frac{c - b}{a}, x_2 = \frac{-c -b}{a}.
            $$

            \begin{proof}
                The solution to this can be shown by dividing it into cases, where $ax + b = -c$ or $ax + b = c$.
            \end{proof}

            \begin{proof}
                This solution can also be shown by squaring both sides to get rid of the absolute value and using quadratic solving methods. I would not prefer it in the first place.
            \end{proof}
        \end{theorem}

        \begin{theorem}[generalise: $\abs{f(x)} = c \geq 0$]
            The solution to this equation is the same as the solutions to $f(x) = \pm c$.
        \end{theorem}

        \begin{theorem}[type $\abs{ax + b} = \abs{cx + d}$]
            The solutions to the equation
            $$
            \abs{ax + b} = \abs{cx + d}
            $$
            is
            $$
            x_{1, 2} = \frac{(cd - ab) \pm 2 (ad - bc)}{a^2 - c^2}.
            $$

            \begin{proof}
                We can show it by squaring both sides getting
                $$
                (ax + b)^2 = (cx + d)^2 \implies (a^2 - c^2)x^2 + 2(ab - cd) x + (b^2 - d^2) = 0.
                $$

                The discriminant will be
                \begin{align*}
                    \Delta &= [2(ab - cd)]^2 - 4 (a^2 - c^2) (b^2 - d^2)\\
                    &= 4 [(a^2 b^2 - 2abcd + c^2 d^2) - a^2b^2 - c^2d^2 + a^2 d^2 + b^2 c^2]\\
                    &= 4 [a^2 d^2 - 2abcd + b^2 c^2]\\
                    &= 4 (ad - bc)^2.
                \end{align*}

                Hence, solutions will be
                \begin{align*}
                    x_{1, 2} &= \frac{- 2 (ab - cd) \pm \sqrt{\Delta}}{2 (a^2 - c^2)}\\
                    &= \frac{(cd - ab) \pm 2 (ad - bc)}{a^2 - c^2}.
                \end{align*}
            \end{proof}

            \begin{proof}
                We can also show this by considering order relationship between $-\frac{b}{a}, - \frac{d}{c}, x$ and expanding absolute values. I would not prefer it in the first place.
            \end{proof}
        \end{theorem}

        \begin{theorem}[type $\abs{f(x)} = \abs{g(x)}$]
            This solution will be the same as $[f(x)]^2 = [g(x)]^2$.
        \end{theorem}

        \begin{theorem}[type $\abs{ax + b} > / \leq c, c \geq 0$]
            Solutions to $\abs{ax + b} > c$ with previous mentioned restrictions will be the same as the solution to
            $$
                (ax + b)^2 - c^2 > 0.
            $$

            I would prefer to square everything if this is a inequality, but we should be careful whether this operation is equivilent or not (will it introduce more solutions? will it ignore certain solutions?)
        \end{theorem}

        \begin{theorem}[type $\abs{ax + b} \leq \abs{cx + d}$]
            Solutions to this will be equivilent to
            $$
            (a^2 - c^2) x^2 + 2(ab - cd)x + (b^2 - d^2) \leq 0.
            $$

            We can also do this by considering the relationship between $x, -\frac{b}{a}, -\frac{d}{c}$, but I think this way is easier.
        \end{theorem}

        \begin{theorem}[graph of $p(x) = k (x-a) (x-b) (x-c), k \neq 0$]
            The graph of $p(x)$ will satisfies the follows:
            $$
                \lim_{x \rightarrow +\infty} p(x) = - \lim_{x \rightarrow -\infty} p(x) = +\infty (\text{if $k$ > 0}), = -\infty (\text{if $k$ < 0}),
            $$
            which shows the trends of the graph of $p(x)$ when it approaches infinity (go further to the left/right) has the same symbol as $k$.

            Furthermore, the intersections of $p(x)$ and $x$-axis will be $ a, b, c$, since intersection with $x$-axis implies $p(x) = 0$ (and they are the only ones due to the fundamental theorem of algebra).
        \end{theorem}

    \section{Indices and surds}
        In this section we will recall the full definition of the power of a positive number.

        \begin{definition}[$a^b$ where $a > 0, b \in \NN$]
            We inductively define it by base case
            $$a^0 = 1,$$
            and $$a^b = a^{b - 1} \cdot a\text{ for } b \geq 1,$$
            in the positive direction, and
            $$a^b = a^{b + 1} \cdot \frac{1}{a}\text{ for } b \leq -1$$
            in the negative direction.
        \end{definition}

        \begin{definition}[$a^b$ where $a > 0, b = \frac{p}{q}, p \in \ZZ, q \in \NN$]
            We define it as
            $$
                a^{b} = a^{\frac{p}{q}} = \sqrt[q]{a^p}.
            $$
        \end{definition}

        \begin{theorem}[calculation properties of powers]
            We have
            \begin{align*}
                a^b \cdot a^c &= a^{b + c},\\
                \frac{a^b}{a^c} &= a^{b - c},\\
                (a^b)^c = (a^c)^b &= a^{bc},\\
                (ab)^c &= a^c b^c,\\
                \left(\frac{a}{b}\right)^c &= \frac{a^c}{b^c}\\
                a^0 &= 1,\\
                a^{-n} &= \frac{1}{a^n},\\
                a^{\frac{1}{n}} &= \sqrt[n]{a},\\
                a^{\frac{m}{n}} &= \left(\sqrt[n]{a}\right)^m = \sqrt[n]{a^m}.
            \end{align*}

            Readers should verify that the previous two definitions are actually well-defined under those properties (because I believe those properties are the reason why we define it as previously defined).
        \end{theorem}

        \begin{theorem}[calculation properties of roots]
            We have
            \begin{align*}
                \sqrt{ab} &= \sqrt{a} \cdot \sqrt{b},\\
                \sqrt{\frac{a}{b}} &= \frac{\sqrt{a}}{\sqrt{b}},\\
                \sqrt{a} \cdot \sqrt{a} &= a.
            \end{align*}
        \end{theorem}

        For those $b \in \QQ'$ (i.e. $b$ is a irrational number), we define it by a limit of rational numbers (Cauchy Sequence).

        To rationalise a fraction, we use the following theorem (in fact, a method, by conjugate roots)
        
        \begin{theorem}[rationalising denominator]
            We have
            $$
                \frac{k}{\sqrt{a}} = \frac{k \sqrt{a}}{a},
            $$
            $$
                \frac{k}{\sqrt{a} - \sqrt{b}} = \frac{k(\sqrt{a} + \sqrt{b})}{a - b},
            $$
            $$
                \frac{k}{\sqrt{a} + \sqrt{b}} = \frac{k(\sqrt{a} - \sqrt{b})}{a - b}.
            $$
        \end{theorem}

    \section{Factors of polynomials}
        \begin{definition}[polynomial]
            A polynomial is an element of the linear space/communicative ring $\PP [x]$.

            Just kidding. A \textbf{polynomial} $p(x)$ can be expressed as a sum
            $$
                p(x) = \sum_{i = 0}^{n} k_i x^i
            $$
            for some non-negative integer $n$ which we call it the degree (DANGER ZONE! degree of the polynomial 0 is negative infinity), and real number $k_i$s which are not all 0. 
        \end{definition}

        \begin{definition}[root]
            A \textbf{root} $x_0$ of a polynomial $p(x)$ satisfies that $p(x_0) = 0$.
        \end{definition}

        \begin{theorem}[factor theorem]
            $x_0$ is a root of $p(x)$ if and only if $(x - x_0)$ is a factor of $p(x)$.

            \begin{proof}
                By basic properties of division, let $p(x) = q(x) (x - x_0) + r(x)$.

                If $x_0$ is a root of $p(x)$, then by definition we have $p(x_0) = r(x) = 0$ which means $p(x)$ has a remainder of $0$ upon division by $x - x_0$.
                
                If $x - x_0$ is a factor of $p(x)$, we have $r(x) = 0$ and $p(x) = q(x) (x - x_0)$. Plugging in $x = x_0$ will give use $p(x) = 0$ hence $x_0$ is a root of $p(x)$.
            \end{proof}
        \end{theorem}

        \begin{theorem}[remainder theorem]
            The remainder of $p(x)$ divided by $(x - x_0)$ will be equal to $p(x_0)$.

            \begin{proof}
                Proof is similar to previous one. The reader should verify so.
            \end{proof}
        \end{theorem}

    \section{Simultaneous equations}
        
        This section does not have a lot to do.

        Two ways of solving simultaneous equations are \textbf{elimination} or \textbf{substitution}.

        There is an advanced way of dealing with Linear Equations (i.e. unknown maximum power of 1) by using matrices, ranks (linear algebra) and Gaussian Elimination. But it's just elimination, using more advanced way to express so.
    
    \section{Logarithmic and exponential functions}
        \begin{definition}[exponential functions]
            An exponential function $f(x)$ has the form of follows
            $$
                f(x) = a^x
            $$
            where $a > 0$ and $a \neq 1$.

            Exponential functions are defined on $\RR$ and has a range of $(0, +\infty)$.
        \end{definition}

        \begin{definition}[logarithm]
            We define the function $\log_a x$ as the inverse of $a^x$ where $a > 0$, $a \neq 1$. In fact,
            $$
                y = a^x \iff x = \log_a y, a > 0, a \neq 1.
            $$
        \end{definition}

        \begin{definition}[logarithm functions]
            A logarithmic function $f(x)$ has the form of follows
            $$
                f(x) = \log_a x
            $$ 
            wherer $a > 0$ and $a \neq 1$.

            In the case of $a = e$, we denote it as
            $$
                f(x) = \ln x,
            $$
            and in the case of $a = 10$, we denote it as
            $$
                f(x) = \lg x.
            $$

            Logarithmic functions are defined on $(0, +\infty)$ and has a range of $\RR$.
        \end{definition}
    
        \begin{theorem}[logarithm calculation basics]
            We have
            $$
                \log_a a = 1, \log_a 1 = 0, \log_a a^x = x, a^{\log_a x} = x.
            $$
        \end{theorem}

        \begin{theorem}[logarithm calculation rules]
            We have
            \begin{align*}
                \log_a (xy) &= \log_a x + \log_a y, \\
                \log_a \left(\frac{x}{y}\right) &= \log_a x - \log_a y,\\
                \log_a (x^m) &= m \log_a x.
            \end{align*}
        \end{theorem}

        \begin{corollary}
            $$
            \log_a \left(\frac{1}{x}\right) = - \log_a x.
            $$
        \end{corollary}

        \begin{theorem}[change of base]
            We have
            $$
                \log_b a = \frac{\log_c a}{\log_c b}.
            $$
        \end{theorem}
        
        \begin{corollary}
            $$
                \log_b a = \frac{1}{\log_a b}.
            $$
        \end{corollary}

        \begin{theorem}[graphs of exponentials]
            For a exponential $a^x$ where $a > 1$, we have
            $$
                \lim_{x \rightarrow +\infty} a^x = +\infty, \lim_{x \rightarrow -\infty} a^x = 0.
            $$

            For a exponential $a^x$ where $a < 1$, we have
            $$
                \lim_{x \rightarrow +\infty} a^x = 0, \lim_{x \rightarrow -\infty} a^x = +\infty.
            $$
        \end{theorem}

        \begin{theorem}[graphs of logarithms]
            For a logarithm $\log_a x$ where $a > 1$, we have
            $$
                \lim_{x \rightarrow 0^+} \log_a x = -\infty, \lim_{x \rightarrow +\infty} \log_a x = +\infty.
            $$ 

            For a logarithm $\log_a x$ where $a < 1$, we have
            $$
                \lim_{x \rightarrow 0^+} \log_a x = +\infty, \lim_{x \rightarrow +\infty} \log_a x = -\infty.
            $$
        \end{theorem}

    \section{Straight line graphs}
        \begin{definition}[straight line]
            A \textbf{straight line} is an equation of the form $y = mx + c$, where $m$ is the \textbf{gradient} and $c$ is the \textbf{$y$-interception}.
        \end{definition}

        \begin{definition}[expression forms]
            Other forms of expressing lines include
            $$
                ax + by + c = 0,
            $$
            and when we know a gradient and a point, we can use
            $$
                (y - y_0) = m(x - x_0).
            $$
        \end{definition}

        \begin{theorem}[mid-point]
            The mid-point of a ling segment with points $A(x_1, y_1)$ and $B(x_2, y_2)$ is
            $$
                M\left(\frac{x_1 + x_2}{2}, \frac{y_1 + y_2}{2}\right).
            $$
        \end{theorem}

        \begin{theorem}[length/distance]
            The distance between two points $A(x_1, y_1)$ and $B(x_2, y_2)$ is
            $$
                \sqrt{(x_1 - x_2)^2 + (y_1 - y_2)^2}.
            $$
        \end{theorem}

        \begin{theorem}[parallel condition]
            Two lines $l_1: y = m_1 x + c_1$ and $l_2: y = m_2 x + c_2$ are parallel if and only if $m_1 = m_2$.
        \end{theorem}

        \begin{theorem}[perpendicular condition]
            Two lines $l_!: y = m_1 x + c_1$ and $l_2: y = m_2 x + c_2$ are perpendicular if and only if $m_1 m_2 = -1$.
        \end{theorem}

        \begin{theorem}[area of a triangle]
            The area of a triangle of three vertices $A(x_1, y_1), B(x_2, y_2), C(x_3, y_3)$ is equal to
            $$
                S = \frac{1}{2} \abs{\begin{vmatrix}
                    1 & 1 & 1\\
                    x_1 & x_2 & x_3\\
                    y_1 & y_2 & y_3
                \end{vmatrix}}
                = \frac{1}{2} \abs{x_1 y_2 + x_2 y_3 + x_3 y_1 - x_2 y_1 - x_3 y_2 - x_1 y_3}.
            $$
        \end{theorem}

    \section{Circular measure}
        \begin{definition}[radians]
            In a circle with radius of $1$, \textbf{$1$ radian} is defined as the angle at the centre when it has a corresponding arc length of $1$.
        \end{definition}

        \begin{corollary}
            $$
                \pi = 180\degree.
            $$
        \end{corollary}

        \begin{corollary}
            Multiply $\frac{180}{\pi}$ to convert degree to radian.
            
            Multiply $\frac{\pi}{180}$ to convert radian to degree.
        \end{corollary}

        \begin{corollary}[arc length]
            $$l = r \theta$$
            where $l$ is arc length, $r$ is radius and $\theta$ is the angle at the centre.
        \end{corollary}

        \begin{corollary}[sector area]
            $$A = \frac{1}{2} r^2 \theta$$
            where $A$ is area of sector, $r$ is radius and $\theta$ is the angle at the center.
        \end{corollary}

    \section{Trigonometry}
        
        \begin{definition}[trig functions, prior knowledge]
            In a right-angle triangle with hypotenuse $r$, angle $\theta$, oppopsite edge $y$ and adjacent edge $x$, we define
            $$
                \sin \theta = \frac{y}{r}, \cos \theta = \frac{x}{r}, \tan \theta = \frac{y}{x}.
            $$
        \end{definition}

        \begin{theorem}[value of important angles]
            $$\sin 0 = 0, \cos 0 = 1, \tan 0 = 0,$$
            $$\sin \frac{\pi}{6} = \frac{1}{2}, \cos \frac{\pi}{6} = \frac{\sqrt{3}}{2}, \tan \frac{\pi}{6} = \frac{1}{\sqrt{3}},$$
            $$\sin \frac{\pi}{4} = \frac{\sqrt{2}}{2}, \cos \frac{\pi}{4} = \frac{\sqrt{2}}{2}, \tan \frac{\pi}{4} = 1,$$
            $$\sin \frac{\pi}{3} = \frac{\sqrt{3}}{2}, \cos \frac{\pi}{3} = \frac{1}{2}, \tan \frac{\pi}{3} = \sqrt{3},$$
            $$\sin \frac{\pi}{2} = 1, \cos \frac{\pi}{2} = 0, \tan \frac{\pi}{2} = \infty.$$
        \end{theorem}

        \begin{definition}[angle]
            An \textbf{angle} is a measure of the rotation of a line $OP$ around $O(0, 0)$ from the positive $x$-direction, with anti-clockwise taken as positive angle and clockwise as negative.
        \end{definition}

        \begin{definition}[trig functions]
            Trigonometric ratios of any angle $\theta$ can be defined as
            $$
                \sin \theta = \frac{y}{r}, \cos \theta = \frac{x}{r}, \tan \theta = \frac{y}{x},
            $$
            where $P(x, y)$ and $r = OP = \sqrt{x^2 + y^2}$.
        \end{definition}

        \begin{theorem}
            All trig functions are positive in the first quadrant. Only $\sin$ is positive in the second quadrant. Only $\tan$ is positive in the third quadrant. Only $\cos$ is positive in the fourth quadrant.
        \end{theorem}

        \newcommand{\cosec}{\mathrm{cosec}}

        \begin{definition}[$\cot, \sec, \cosec$]
            We define the three extra trig functions as follows:
            $$
                \cot \theta = \frac{x}{y}, \sec \theta = \frac{r}{x}, \cosec \theta = \frac{r}{y}.
            $$
        \end{definition}

        \begin{theorem}[trig identities]
            The following trig identities are very basic and important:
            $$
                \cosec \theta = \frac{1}{\sin \theta}, \sec \theta = \frac{1}{\cos \theta}, \cot \theta = \frac{1}{\tan \theta},
            $$
            $$
                \sin^2 \theta + \cos^2 \theta = 1, 1 + \tan^2 \theta = \sec^2 \theta, \cot^2 \theta + 1 = \cosec^2 \theta,
            $$
            $$
                \tan \theta = \frac{\sin \theta}{\cos \theta}, \cot \theta = \frac{\cos \theta}{\sin \theta}.
            $$

            In this case I feel I have to make the following clear while dealing with $\infty$ of $\tan, \cot, \sec, \cosec$. We treat $0 \cdot \infty = 1, 0 = \frac{1}{\infty}, \infty = \frac{1}{0}$. DANGER! Informal use of notation.
        \end{theorem}

        \begin{definition}[periodic function]
            For a function $f: A \rightarrow B$, if for a certain real number $T$, we have $x \in A \implies x + T \in A$, and
            $$f(x) = f(x + T)$$
            for all $x \in A$, we say $T$ is a \textbf{period} of the function $f$. We say $T$ is a \textbf{minimum positive period} if $T$ is the smallest positive number which is a period of that function.
        \end{definition}

        \begin{theorem}[trig as functions]
            The function $y = \sin x$ has a domain of $\RR$ and a range of $[-1, 1]$. It has a minimum positive period of $2\pi$, and an amplitude of $1$. Restricting its domain onto $[-\frac{\pi}{2}, \frac{\pi}{2}]$, its inverse, defined from $[-1, 1]$ to $[-\frac{\pi}{2}, \frac{\pi}{2}]$ is denoted as $\arcsin x$.

            The function $y = \cos x$ has a domain of $\RR$ and a range of $[-1, 1]$. It has a minimum positive period of $2\pi$, and an amplitude of $1$. Restricting its domain to $[0, \pi]$, its inverse, defined from $[-1, 1]$ onto $[0, \pi]$ is denoted as $\arccos x$.

            The function $y = \tan x$ has a domain of $\RR \setminus \{\frac{\pi}{2} + k\pi \mid k \in \ZZ\}$ (all real numbers except for $\{\ldots, -\frac{\pi}{2}, \frac{\pi}{2}, \frac{3\pi}{2}, \ldots\}$), and a range of $\RR$. It has a minimum positive period of $\pi$. Restricting its domain to $(-\frac{\pi}{2}, \frac{\pi}{2})$, its inverse, defined from $\RR$ onto $(-\frac{\pi}{2}, \frac{\pi}{2})$ is denoted as $\arctan x$.

            Notice that since $\forall k \in \ZZ$, we have
            $$
                \lim_{x \rightarrow k\pi + \frac{\pi}{2}} \tan x = \infty,
            $$
            we know that the lines $x = k\pi + \frac{\pi}{2}, k \in \ZZ$ are the vertical asymptotes for $y = \tan x$.

            The function $y = \cot x$ has a domain of $\RR \setminus \{k\pi \mid k \in \ZZ\}$ (all real numbers except for $\{\ldots, -\pi, 0, \pi, 2\pi, \ldots\}$), and a range of $\RR$. It has a minimum positive period of $\pi$.

            Notice that since $\forall k \in \ZZ$, we have
            $$
                \lim_{x \rightarrow k\pi} \cot x = \infty,
            $$
            we know that the lines $x = k\pi, k \in \ZZ$ are the vertical asymptotes for $y = \cot x$.

            The function $y = \sec x$ has a domain of $\RR \setminus \{\frac{\pi}{2} + k\pi \mid k \in \ZZ\}$ and a range of $(-\infty, -1] \cup [1, +\infty)$. It has a minimum positive period of $2\pi$.

            The function $y = \csc x$ has a domain of $\RR \setminus \{k\pi \mid k \in \ZZ\}$ and a range of $(-\infty, -1] \cup [1, +\infty)$. It has a minimum positive period of $2\pi$.
        \end{theorem}

        \begin{theorem}[trig function manipulations]
            The function $y = a \sin bx + c$ has an amplitude of $a$, a period of $\frac{2\pi}{b}$ and is translated upwards by $c$ units.

            The function $y = a \cos bx + c$ has an amplitude of $a$, a period of $\frac{2\pi}{b}$ and is translated upwards by $c$ units.

            The function $y = a \tan bx + c$ stretches the graph vertically by a factor of $a$, has a period of $\frac{\pi}{b}$ and is translated upwards by $c$ units.
        \end{theorem}


    \section{Differentiation and integration}

        

    \section{Vectors in two dimensions}

        

    \section{Permutations and combinations}

        

    \section{Series}

        

    \section*{Afterwords}

        

\end{document}