\documentclass[8pt]{article}
\usepackage{allan-eason}

\usepackage{siunitx}

\usepackage[english]{babel}

\theoremstyle{remark}

\newtheorem{theorem}{Theorem}[section]
\newtheorem{lemma}[theorem]{Lemma}
\newtheorem{corollary}[theorem]{Corollary}
\newtheorem{definition}[theorem]{Definition}
\newtheorem{example}[theorem]{Example}

\newtheorem*{remark}{Remark}

\usetikzlibrary{positioning}
\usetikzlibrary{svg.path}

\graphicspath{ {./images/} }


\newcommand{\Date}{\today}
\newcommand{\Name}{GCSE Maths Knowledge Sheet}
\newcommand{\Subname}{Eason's Mathematics Toolbox}
\newcommand{\Title}{\Name\\ \Large{\Subname}}

\newcommand{\Author}{Yicheng Shao}

\author{\Author}
\title{\Title}
\date{Version 1. \Date}

\lhead{\Name}

\begin{document}

	\maketitle

	\tableofcontents

    \section*{What is this and why this?}

        This document assumes prior knowledge in CIE IGCSE Mathematics.
    
    \section{Functions}

        \begin{definition}[function, domain, image]
            A \textbf{fucntion} $f: A \rightarrow B$ is defined as a mapping which maps each element in $A$ to exactly one element in $B$. Basically, a function is an operation on a thing which definitely produces another thing.

            We call $A$ the \textbf{domain} (the set which this function can operate on). (And $B$ the co-domain.)

            We define the set
            $$
                \{f(x) \mid x \in A\}
            $$
            as the \textbf{range} of the function, which is all the outputs of the function.

            At this stage, $B$ will be $\RR$ and $A$ will be a subset of $\RR$.
        \end{definition}

        \begin{definition}[one-to-one, many-to-one]
            We call a function $f$ \textbf{one-to-one}, or injective, when
            $$
                f(x_1) = f(x_2) \implies x_1 = x_2.
            $$
            
            This means that each output a function will produce can only appear by opeating on exactly one element.

            If a function is not one-to-one, we call it \textbf{many-to-one}.
        \end{definition}

        \begin{definition}[function notations]
            The result that $f$ maps an element of the domain $x$ to is denoted as $f(x)$. As an example, if function $f$ maps $x$ to $\sin x$, then the following are equivilant:
            \begin{enumerate}
                \item $f(x) = \sin x$,
                \item $f: x \mapsto \sin x$.
            \end{enumerate}
        \end{definition}

        \begin{definition}[inverse]
            A function's inverse, denoted as $f^{-1}(x)$, is defined from the range of $f(x)$ to the domain of $f(x)$, and satisfies that:
            $$
                f^{-1}(f(x)) = f(f^{-1}(x))= x.
            $$
        \end{definition}

        \begin{theorem}[unique inverse]
            If the inverse of a function exists, then it is unique.
        \end{theorem}

        \begin{theorem}[condition for existance of inverse]
            The inverse of a function exists if and only if such function is one-to-one.
        \end{theorem}

        \begin{remark}
            The previous theorem is true if and only if such inverse is defined from the range. If such inverse is defined from the co-domain then we also require the function to surjective (i.e. range equals to co-domain) hence bijective. This is a very useful concept (isomorphism)!
        \end{remark}

        \begin{theorem}[inverse graphs]
            The graph of a inverse of a function and the function itself is symmetric by the line $y = x$.
        \end{theorem}

        \begin{definition}[composite]
            The composite of $f$ with $f$ denoted as $f^2$ is defined as follows:
            $$
            f^2(x) = f(f(x)).
            $$
        \end{definition}

        \begin{definition}[modulus]
            The graph of $\abs{f(x)}$ and $f(x)$ has a relationship as follows:

            The graph of $\abs{f(x)}$ reflects the part of the graph of $f(x)$ below the $x$ axis with regards to the $x$ axis (basically flip it up).
        \end{definition}

    \section{Quadratic}
        \begin{definition}[quadratic]
            A quadratic function $f$ is defined as an element of $\PP[x]$ where $\deg f(x) = 2$.

            Just kidding. A quadratic function $f$ is defined as
            $$
            f(x) = ax^2 + bx + c
            $$
            where $a \neq 0$.
        \end{definition}

        \begin{theorem}[extremum property]
            A quadratic function $f(x)$ has a maximum if and only if $a < 0$, and it has a minimum if and only if $a > 0$. The turning point (extremum point in this case) of a quadratic is
            $$
                \left(-\frac{b}{2a}, \frac{4ac - b^2}{4ac}\right).
            $$

            \begin{proof}
                We can show this by \textbf{completing the square}.
                \begin{align*}
                    ax^2 + bx + c &= a \left(x^2 + \frac{b}{a} x\right) + c\\
                    &= a \left(x^2 + 2 \cdot \frac{b}{2a} \cdot x \right) + c\\
                    &= a \left[x^2 + 2 \cdot \frac{b}{2a} \cdot x + \left(\frac{b}{2a}\right)^2\right] - a \cdot \left(\frac{b}{2a}\right)^2 + c\\
                    &= a \cdot \left(x + \frac{b}{2a}\right)^2 - \frac{b^2}{4a} + c.
                \end{align*}

                If $a > 0$, then we have
                \begin{align*}
                    ax^2 + bx + c \geq -\frac{b^2}{4a} + c,
                \end{align*}
                where the equal sign holds if and onlly if $x = - \frac{b}{2a}$.

                Similar argument holds for $a < 0$.
            \end{proof}

            \begin{proof}
                We can also show this by \textbf{differentiation}.
            \end{proof}
        \end{theorem}

        \begin{theorem}[roots]
            The roots (solutions) to the quadratic will be
            $$
                x_{1, 2} = \frac{-b \pm \sqrt{b^2 - 4ac}}{2a}.
            $$
        \end{theorem}

        \begin{theorem}[discriminant]
            The \textbf{discriminant} for the quadratic $ax^2 + bx + c$ is defined as
            $$
                \Delta = b^2 - 4ac.
            $$

            When $\Delta > 0$, the quadratic has two distinct real roots; when $\Delta = 0$, the quadratic has two equal real roots; when $\Delta < 0$, the quadratic has two complex roots which are complex conjugate of each other (i.e. sum to a real number).
        \end{theorem}

        \begin{theorem}[quadratic and a line]
            The intersections for a quadratic and a line (which is not perpendicular to the $x$ axis) can be found by equating their equations and solve the corresponding equation (which is a quadratic).
        \end{theorem}

        \begin{definition}[intervals]
            We define the intervals as follows:
            \begin{align*}
                (a, b) &= \{x \mid a < x < b\},\\
                (a, b] &= \{x \mid a < x \leq b\},\\
                [a, b) &= \{x \mid a \leq x < b\},\\
                [a, b] &= \{x \mid a \leq x \leq b\},\\
                (a, +\infty) &= \{x \mid x > a\},\\
                [a, +\infty) &= \{x \mid x \geq a\},\\
                (-\infty, b) &= \{x \mid x < b\},\\
                (-\infty, b] &= \{x \mid x \leq b\},\\
                (-\infty, +\infty) &= \RR.
            \end{align*}
        \end{definition}

        \begin{theorem}[quadratic inequalities]
            A quadratic inequality can be solved by finding the two solutions (known as \textbf{critical values}).

            For the quadratic $f(x) = ax^2 + bx + c$ where $a > 0$ ($a < 0$ can be considered similarly),
            \begin{enumerate}
                \item $\Delta > 0$. Let the two roots be $x_1$ and $x_2$.
                
                The solution set to $f(x) > 0$ is
                $$
                    (-\infty, x_1) \cup (x_2, +\infty).
                $$

                The solution set to $f(x) < 0$ is
                $$
                    (x_1, x_2).
                $$

                \item $\Delta = 0$. Let the root be $x_r$.
                
                The solution set to $f(x) > 0$ is
                $$
                    (-\infty, x_r) \cup (x_r, +\infty).
                $$

                The solution set to $f(x) < 0$ is $\emptyset$.

                \item $\Delta < 0$. The solution set to $f(x) > 0$ is $\RR$ and the solution set to $f(x) < 0$ is $\emptyset$.
            \end{enumerate}
        \end{theorem}

    \section*{Afterwords}
        

\end{document}